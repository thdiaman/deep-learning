\documentclass[10pt,twoside,a4paper,openany]{memoir}

\setlrmarginsandblock{2.35cm}{2.35cm}{*}
\setulmarginsandblock{2.35cm}{2.35cm}{*}
\renewcommand{\baselinestretch}{1.05}
\checkandfixthelayout

\makechapterstyle{box}{
  \renewcommand*{\printchaptername}{}
  \renewcommand*{\printchapternum}{}
  \renewcommand*{\printchaptertitle}[1]{\vspace{-2.5cm}\chaptitlefont\thechapter.\ ##1\vspace{-1cm}}
}
\chapterstyle{box}

\setsecnumdepth{subsubsection}

\usepackage{graphicx}

\usepackage{listings}


% Default fixed font does not support bold face
\DeclareFixedFont{\ttb}{T1}{txtt}{bx}{n}{9.5} % for bold
\DeclareFixedFont{\ttm}{T1}{txtt}{m}{n}{10}  % for normal

% Custom colors
\usepackage{color}
\definecolor{deepblue}{rgb}{0,0,0.5}
\definecolor{deepred}{rgb}{0.6,0,0}
\definecolor{deepgreen}{rgb}{0,0.5,0}
\lstset{
language=Python,
basicstyle=\ttm,
otherkeywords={self},             % Add keywords here
keywordstyle=\ttb\color{deepblue},
emph={MyClass,__init__},          % Custom highlighting
emphstyle=\ttb\color{deepred},    % Custom highlighting style
stringstyle=\color{deepgreen},
frame=tb,                         % Any extra options here
showstringspaces=false,
breaklines=true
%basicstyle=\small%\linespread{1.15}\sffamily
}

\usepackage[hidelinks,bookmarksopen=true,bookmarksopenlevel=-1]{hyperref}

\usepackage{qrcode}

\let\cleardoublepage\clearpage

\makeatletter
\def\maketitle{%
  \null
  \thispagestyle{empty}%
  \vfill

  \begin{center}
    \normalfont
    \centering

    \begin{minipage}{\linewidth}
    \centering
    \includegraphics[scale=0.3]{images/box-icon.png}
    \end{minipage}

    \vskip 1cm
    
    {\huge\@title\par}%
    \hrulefill\par
    \vskip 1cm

    {\LARGE\raggedleft Manual by \@author\par}%
  \end{center}%
  \vfill
  \null
  \clearpage
  }
\makeatother
\author{Themistoklis Diamantopoulos}
\title{Deep Learning with Python Keras}
\date{}

\makepagestyle{headings}
\makeevenhead{headings}{\thepage}{}{DEEP LEARNING WITH PYTHON KERAS}
\makeoddhead{headings}{DEEP LEARNING WITH PYTHON KERAS}{}{\thepage}

\begin{document}

\maketitle

\frontmatter

\null
\thispagestyle{empty}%
\vfill

\begin{flushleft}
\textit{Deep Learning with Python Keras}

Notes for a tutorial compiled by Themistoklis Diamantopoulos

\vskip 0.2cm

For more information check website:

\url{https://thdiaman.github.io/deeplearning/}

or check the following QR code:

\vskip 0.2cm

\XeTeXLinkBox{\qrcode[height=25mm]{https://thdiaman.github.io/deeplearning/}}

\bigskip

THE MATERIAL USED IN THIS TUTORIAL IS GATHERED FROM A SET OF DIFFERENT SOURCES THAT ARE REFERRED IN THE REFERENCES SECTION

\vskip 0.2cm
ALL RIGHTS RESERVED TO THE ORIGINAL OWNERS

\vskip 0.2cm
THIS DOCUMENT IS ONLY TO SERVE AS A MANUAL FOR THE CONTENT OF THE TUTORIAL

\vskip 0.2cm
COPYRIGHT 2018

\end{flushleft}



\mainmatter

\chapter{Optical Character Recognition}

\section{Solution using Fully Connected Neural Network}
\subsection{Training}
\lstinputlisting{../OCR/training.py}

\subsection{Testing}
\lstinputlisting{../OCR/testing_metrics.py}

\section{Solution using Convolutional Neural Network}
\subsection{Training}
\lstinputlisting{../OCR/training_cnn.py}

\subsection{Testing}
\lstinputlisting{../OCR/testing_metrics_cnn.py}

\chapter{Image Recognition}

\section{Solution using Convolutional Neural Network}
\subsection{Training}
\lstinputlisting{../ImageRecognition/training.py}

\subsection{Testing}
\lstinputlisting{../ImageRecognition/testing.py}

\section{Solution using Bottleneck Features on VGG16}
\subsection{Training}
\lstinputlisting{../ImageRecognition/training_vgg16.py}

\subsection{Testing}
\lstinputlisting{../ImageRecognition/testing_vgg16.py}

\chapter{Text Classification}
\section{Solution using Fully Connected Neural Network}
\subsection{Training}
\lstinputlisting{../TextClassification/training.py}

\subsection{Testing}
\lstinputlisting{../TextClassification/testing.py}

\newpage
\section{Solution using Convolutional Neural Network}
\subsection{Training}
\lstinputlisting{../TextClassification/training_cnn.py}

\subsection{Testing}
\lstinputlisting{../TextClassification/testing_cnn_lstm.py}

\section{Solution using Recurrent Neural Network}
\subsection{Training}
\lstinputlisting{../TextClassification/training_lstm.py}

\subsection{Testing}
Same as Testing for Solution using Convolutional Neural Network

\chapter{Text Generation}
\section{Solution using Recurrent Neural Network}
\subsection{Training}
\lstinputlisting{../TextGeneration/training_rnn.py}

\subsection{Testing}
\lstinputlisting{../TextGeneration/testing_rnn.py}

\chapter{Game Playing}
\section{Catch}
\lstinputlisting{../CatchGame/training.py}

\newpage
\section{Maze}
\lstinputlisting{../MazeGame/training.py}

\backmatter

\renewcommand*{\printchaptertitle}[1]{\chaptitlefont#1}
\chapter*{References}
This chapter contains any references used to create this tutorial.
In specific, it contains sources for the different source
code parts of each section/example of this tutorial.

\section{OCR}
\begin{itemize}
\item MNIST in Keras: \href{https://github.com/wxs/keras-mnist-tutorial/blob/master/MNIST\%20in\%20Keras.ipynb}{https://github.com/wxs/keras-mnist-tutorial/blob/master/MNIST\%20in\%20Kera s.ipynb}
\item A simple 2D CNN for MNIST digit recognition: \href{https://towardsdatascience.com/a-simple-2d-cnn-for-mnist-digit-recognition-a998dbc1e79a}{https://towardsdatascience.com/a-simple-2d-cnn-for-mnist-digit-recognition-a998dbc1e79a}
\end{itemize}

\section{Image Recognition}
\begin{itemize}
\item Building powerful image classification models using very little data: \href{https://blog.keras.io/building-powerful-image-classification-models-using-very-little-data.html}{https://blog.keras.io/buildi ng-powerful-image-classification-models-using-very-little-data.html}
\item Object classification using CNN \& VGG16 Model (Keras and Tensorflow): \href{https://www.slideshare.net/LalitJain29/object-classification-using-cnn-vgg16-model-keras-and-tensorflow}{https://www.slidesh are.net/LalitJain29/object-classification-using-cnn-vgg16-model-keras-and-tensorflow}
\end{itemize}

\section{Text Classification}
\begin{itemize}
\item Classifying Tweets with Keras and TensorFlow: \href{https://vgpena.github.io/classifying-tweets-with-keras-and-tensorflow/}{https://vgpena.github.io/classifying-tweets-with-keras-and-tensorflow/}
\item Practical Neural Networks with Keras: Classifying Yelp Reviews: \href{http://www.developintelligence.com/blog/2017/06/practical-neural-networks-keras-classifying-yelp-reviews/}{http://www.developintelligen ce.com/blog/2017/06/practical-neural-networks-keras-classifying-yelp-reviews/}
\end{itemize}

\section{Text Generation}
\begin{itemize}
\item Question answering on the Facebook bAbi dataset using recurrent neural networks and 175 lines of Python + Keras: \href{http://smerity.com/articles/2015/keras_qa.html}{http://smerity.com/articles/2015/keras\_qa.html}
\end{itemize}

\section{Neural Doodle \& Style Transfer}
\begin{itemize}
\item Neural Style Transfer In Keras: \href{https://markojerkic.com/style-transfer-keras/}{https://markojerkic.com/style-transfer-keras/}
\end{itemize}

\section{Game Playing}
\begin{itemize}
\item Deep reinforcement learning: where to start: \href{https://medium.freecodecamp.org/deep-reinforcement-learning-where-to-start-291fb0058c01}{https://medium.freecodecamp.org/deep-reinforce ment-learning-where-to-start-291fb0058c01}
\item Deep Reinforcement Learning for Maze Solving: \href{http://www.samyzaf.com/ML/rl/qmaze.html}{http://www.samyzaf.com/ML/rl/qmaze.html}
\end{itemize}



\end{document}
